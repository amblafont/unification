% \usepackage{subcaption}
% for subfigures/subtables
%\usepackage[caption=false,font=footnotesize]{subfig}
\usepackage{tikz-cd}
\usepackage{adjustbox}
\usepackage{quiver}
\newcommand{\Ker}{\mathrm{Ker}}
\usepackage{mathpartir}
\usepackage{fullshort}
\usepackage{color}

% to comment for the short version
\setboolean{fullpaper}{true}

\newcommand{\pullbackcorner}[1][dl]{\save*!/#1-1pc/#1:(-1,1)@^{|-}\restore}
\newcommand{\commentaire}[1]{}
\usepackage{cmll}

\makeatother
\newcommand{\coeqr}[2]{\ar@<+.5ex>[r]^-{#1}\ar@<-.5ex>[r]_-{#2}}
\makeatletter
\newcommand{\hautr}[1]{\ar[r]^-{#1}}
\newcommand{\hautl}[1]{\ar[l]_-{#1}}


\newcommand{\vecif}[2]{\ifthenelse{\equal{#1}{}}{#2}{\vec{#2}}}

%\newref{rem}{refcmd={Remark~\ref{#1}}}
\newrefformat{rem}{Remark~\ref{#1}}
%\newref{def}{refcmd={Definition~\ref{#1}}}
\newrefformat{def}{Definition~\ref{#1}}
%\newref{assu}{refcmd={Property~\ref{#1}}}
\newrefformat{assu}{Property~\ref{#1}}
%\newref{cor}{refcmd={Corollary~\ref{#1}}}
%\newref{lem}{refcmd={Lemma~\ref{#1}}}
%\newref{prop}{refcmd={Proposition~\ref{#1}}}
%\newref{nota}{refcmd={Notation~\ref{#1}}}
%\newref{tab}{refcmd={Table~\ref{#1}}}
%\newref{subsec}{refcmd={Section~\S\ref{#1}}}

%%%
%% from the paper quantum lambda
\usepackage{proof}

\newcommand{\ass}{:=}
\newcommand{\abs}[1]{|#1|}              
\newcommand{\homof}[2]{#1(#2)}          
\newcommand{\seq}{\subseteq}    
\newcommand{\such}{\,\,|\,\,}           
\newcommand{\pc}{\mathbin{;}}           
\newcommand{\id}{{\textrm{\rm id}}}     
\newcommand{\N}{\mathbb{N}}                     
\newcommand{\categ}[1]{\textrm{\bf #1}} 
\newcommand{\FV}{{\rm FV}}                      
\newcommand{\bang}{{!}}
\newcommand{\tensor}{\otimes}
\newcommand{\bor}{\ \ \rule[-.75ex]{.01in}{2.75ex}\ \ }  
%\newcommand{\C}{\mathbb{C}}                     
\newcommand{\Rp}{{\mathbb{R}^+}}                     
\newcommand{\sqleq}{\sqsubseteq}
\newcommand{\sqgeq}{\sqsupseteq}
\newcommand{\bit}{{\bf bit}}
\newcommand{\qubit}{{\bf qubit}}
\newcommand{\xmatrix}[4]{{\renewcommand{\arraystretch}{#1}%
\arraycolsep=#2ex\left(\begin{array}{#3}#4\end{array}\right)}}
\newcommand{\zmatrix}{\xmatrix{0.8}{0.8}}
\newcommand{\zzmatrix}[2]{{\textrm{\scriptsize$\zmatrix{#1}{#2}$}}}
\newcommand{\CPM}{\textrm{\bf CPM}}
\newcommand{\CPMs}{\textrm{\bf CPM}_s}
\newcommand{\inv}{^{-1}}
\newcommand{\freecat}[1][C]{\ensuremath{\categ{#1}^{\oplus}}}  
\newcommand{\cpms}{\categ{CPMs}} 
\newcommand{\ccpms}{\ensuremath{\overline{\categ{CPMs}}}} 
\newcommand{\M}{{\cal M}}
\newcommand{\supp}[1]{\vert #1\vert}   
\newcommand{\eq}[2][]{\mathop{eq^{#2}_{#1}}\nolimits} 
\newcommand*{\mystackrel}[2]{%
\stackrel{\raise-2pt\hbox{$\scriptstyle\!#1$}}{#2}}
\newcommand*{\redto}[1][]{\mystackrel{#1}{\rightarrow}}
\newcommand*{\xredto}[1][]{%
\xrightarrow{\raisebox{-2pt}{$\scriptstyle\!{#1}$}}}
\newcommand{\qfin}[1]{\mathfrak{#1}} 
\newcommand{\web}[1]{\vert{#1}\vert} 
\newcommand{\swap}[1][]{\sigma_{#1}} 
\newcommand{\proj}[1]{\pi^{#1}} 
\newcommand{\inj}[1]{\iota^{#1}} 
\newcommand{\unit}{\mathbf{1}} 
\newcommand{\Cl}{\mathrm{Cl}} 
\newcommand{\Val}{\mathrm{Val}} 
\newcommand{\apprv}{\mathrel{\triangleleft}} 
\newcommand{\contr}{\mathtt{c}} 
\newcommand{\der}{\mathtt{d}} 
\newcommand{\weak}{\mathtt{w}} 
\newcommand{\dig}{\mathtt{dig}} 
\newcommand{\bierman}{\mathtt{m}} 
\newcommand{\cmatrix}[1]{\Lambda(#1)} 
\newcommand{\injl}[2][]{{\mathtt{in}_\ell^{#1}}~{#2}} 
\newcommand{\injr}[2][]{{\mathtt{in}}_r^{#1}~{#2}} 
\newcommand{\fprod}[2]{\langle #1\rangle_{#2}}
\newcommand{\fcoprod}[2]{[#1]_{#2}}
\newcommand{\pdistr}{{\mathtt{distr}}} 
\newcommand{\symgroup}{S}	
\newcommand{\Lowner}{L\"owner}
\newcommand{\tr}{\mathop{\mathrm{\rm tr}}\nolimits}
\newcommand{\cs}{c}
\newcommand{\down}{{\downarrow}}
\newcommand{\define}[1]{{\em #1}}
\newcommand{\punit}{\mathtt{skip}}
\newcommand{\tensterm}[2]{{{#1}\tensor{#2}}}
\newcommand{\lettensterm}[4]{{\mathtt{let}~{#1}%
\tensor{#2}~=~{#3}~\mathtt{in}~{#4}}}
\newcommand{\letunitterm}[2]{{#1}\mathtt{;}{#2}}
\newcommand{\ttrue}{\mathtt{tt}}
\newcommand{\ffalse}{\mathtt{ff}}
\newcommand{\iftermx}[3]{{{\mathtt{if}}~{#1}~\mathtt{then}~{#2}~%
\mathtt{else}~{#3}}}
\newcommand{\match}[5]{{{\mathtt{match}}~{#1}~{\mathtt{with}}%
~({#2:#3}\mid{#4:#5})}}
\newcommand{\letrec}[4]{{{\mathtt{letrec}}~{#1}\,{#2}={#3}~{\mathtt{in}}~{#4}}}
\newcommand{\letrecn}[5]{{{\mathtt{letrec}}^{#1}~{#2}%
\,{#3}={#4}~{\mathtt{in}}~{#5}}}
\newcommand{\nil}{{\mathtt{nil}}}
\newcommand{\cons}[3][]{{{#2}\,{\mathtt :}{\mathtt :}^{#1}\,{#3}}}
\newcommand{\splitlist}[1][]{{\mathtt{split}^{#1}}}
\newcommand{\meas}{{\mathtt{meas}}}
\newcommand{\new}{{\mathtt{new}}}
\newcommand{\loli}{\multimap}
\newcommand{\tlist}[1]{{{#1}^\ell}}
\newcommand{\sumtype}{\oplus}
\newcommand{\tunit}{{1}}
\newcommand{\arity}[1]{d^{#1}} 
\newcommand{\symm}[1]{G^{#1}} 
\newcommand{\entail}{\vdash}
\newcommand{\am}[1]{{[}{#1}{]}}
\newcommand{\qarray}{q} 
\newcommand{\qlist}{\ell} 
\newcommand{\ket}[1]{{|{#1}\rangle}}
\newcommand{\dual}[1]{{#1}^{\perp}} 
\newcommand{\denot}[1]{{\llbracket #1 \rrbracket}}

\newcommand{\inferruletwo}[3]{\inferrule{#1 \\ #2}{#3}}

\newref{rem}{refcmd={Remark~\ref{#1}}}
\newref{def}{refcmd={Definition~\ref{#1}}}
\newref{assu}{refcmd={Property~\ref{#1}}}
\newref{cor}{refcmd={Corollary~\ref{#1}}}
\newref{lem}{refcmd={Lemma~\ref{#1}}}
\newref{prop}{refcmd={Proposition~\ref{#1}}}
\newref{nota}{refcmd={Notation~\ref{#1}}}
\newref{tab}{refcmd={Table~\ref{#1}}}
\newref{subsec}{refcmd={Section~\S\ref{#1}}}
\newref{app}{refcmd={Appendix~\S\ref{#1}}}
\newref{ex}{refcmd={Example~\ref{#1}}}
\newref{not}{refcmd={Notation~\ref{#1}}}

\newcommand\mydots{\makebox[1em][c]{.\hfil.\hfil.}}


%% acmart
%% Journal information
%% Supplied to authors by publisher for camera-ready submission;
%% use defaults for review submission.
\acmJournal{PACMPL}
\acmVolume{1}
\acmNumber{POPL} % CONF = POPL or ICFP or OOPSLA
\acmArticle{1}
\acmYear{2025}
\acmMonth{1}
\acmDOI{} % \acmDOI{10.1145/nnnnnnn.nnnnnnn}
\startPage{1}

\setcopyright{none}
% \setcopyright{acmcopyright}
% \copyrightyear{2024}

% %\citestyle{acmauthoryear}
% \acmDOI{XXXXXXX.XXXXXXX}

%% These commands are for a PROCEEDINGS abstract or paper.
% \acmConference[Submitted to LICS '24]{Submitted to LICS
  % '24}{2024}{Tallinn}
%%
%%  Uncomment \acmBooktitle if the title of the proceedings is different
%%  from ``Proceedings of ...''!
%%
%%\acmBooktitle{Woodstock '18: ACM Symposium on Neural Gaze Detection,
%%  June 03--05, 2018, Woodstock, NY}
%\acmPrice{gratos}
% \acmISBN{978-1-4503-XXXX-X/18/06}


\newcommand{\cal}[1]{\mathcal{#1}}
\usepackage{agda}
%% vskip breaks the latex when occuring in subfigure
%% with the belwo redefinition, it works by prefexing by AgdaNoSpaceAroundCode
% \renewcommand{\Agda@NewlineWithVerticalSpace}[1]{\parskip=0pt\parindent=0pt\par%
%   \ifthenelse{\equal{#1}{0pt}}{}{\vskip #1}%
%   \noindent}
\usepackage{catchfilebetweentags}
\usepackage[ensuremath]{ebutf8}
% \usepackage{newunicodechar}
% \newunicodechar{λ}{\ensuremath{\mathnormal\lambda}}
% \newunicodechar{←}{\ensuremath{\mathnormal\from}}
% \newunicodechar{→}{\ensuremath{\mathnormal\to}}

\newcommand{\agdalatexdir}{latex-agda}
\newcommand{\agdacode}[1]{%
\ExecuteMetaData[\agdalatexdir/lc.tex]{#1}%
\ExecuteMetaData[\agdalatexdir/lib.tex]{#1}%
\ExecuteMetaData[\agdalatexdir/Common.tex]{#1}%
\ExecuteMetaData[\agdalatexdir/main.tex]{#1}}

% \newcommand{\agdacodenovspace}[1]{%
% \AgdaNoSpaceAroundCode{}
% \agdacode{#1}
% \AgdaSpaceAroundCode{}}


\newcommand{\texHighlighted}[1]{\colorbox{blue!10}{\ensuremath{#1}}}
\newcommand{\texHighlightedFunction}[1]{\texHighlighted{\textrm{\textcolor{black}{#1}}}}

\DeclareRobustCommand{\AgdaFormat}[2]{%
  \ifthenelse{
    \equal{#1}{commonPositions} \OR
    \equal{#1}{commonValues} \OR
    \equal{#1}{equaliser} \OR
    \equal{#1}{pullback}
  }{\texHighlightedFunction{#1}}{%
#2}}
%     {\ifthenelse{\equal{#1}{Tm}}{\{#1}}%
% {#2}%
%   }}}
